\chapter{结论和进一步工作}

\section{工作总结}

随着社交网络的蓬勃发展,人们的线下生活越来越多地反映到社交媒体发布的信息中。通过对社交媒体中用户表达的日常活动进行发掘,能够发现关于活动的许多知识,诸如地域偏好,时间分布,活动之间的概念联系。同时,我们也能够了解到用户的个人喜好,行为模式,对用户进行个性化推荐。例如,用户来到一个新的城市,可以根据他在社交网络中的历史信息以及其他用户在这个城市参与的活动,像他推荐感兴趣的活动项目。

本文对相关研究进行挖掘,确定了三个研究内容:(1)如果精确地从社交媒体中抽取出日常活动的概念表示?(2)针对用户发布的信息,如何抽取出相关活动及相关属性,如地点、时间、情感?(3)如何构建活动之间的关系层次?对于第一个问题,我将抽取问题转换为分类问题,使用神经网络语言模型word2vec获得个短语的语义向量表示,并训练分类模型,取得了较好的分类精度。第二个问题中,我借助哈工大LTP工具,并利用微博自身元信息和POI数据,构建出活动的实例。第三个问题,我参考\cite{wang2013phrase}中提出的框架,构建了活动的层次关系,并对微博时间序列进行统计,得到活动序列关系。

基于以上三点研究,我构建了基于SAE(social analytic engine)的演示系统,对于用户的查询,将活动相关的知识可视化。

\section{局限和进一步工作}

本体学习已经有很长的历史,但在社交媒体中挖掘活动相关知识很少有前人的工作。本文在这一课题上进行尝试性的研究,并构建了初步的原型系统,但由于时间和能力所限,还有不少局限性。在将来,需要在以下方面进行改进:
\begin{enumerate}
	\item 考虑用户在社交网络中的互动关系。用户的好友和网络结构可以提供关于用户的很多信息。例如,用户在进行活动时提到了他的好友,可能隐含其好友也在参与了这项活动,而在当前的工作中,用户作为一个个体存在,并没有考虑到用户间的互动。
	\item 对活动关系的建模比较粗糙。活动之间的关系式多样的,并不能简单的用上下位、follow-up来概括。我们需要构建活动知识的图,而不是简单的层次。
	\item 本文现阶段的工作,仅在社交媒体的数据进行挖掘。如果能现有的知识库,如地点、人物、书籍等结合,能对活动抽取和关系建模有很大的帮助。
\end{enumerate}

希望本文的研究可以继续发展,成为成熟的系统,为用户提供服务。