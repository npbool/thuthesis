\chapter{结论}

\section{工作总结}
随着社交网络和移动互联网的蓬勃发展,人们的线下生活越来越多地反映到社交网络发布的信息中。通过对社交数据中日常活动信息的挖掘,我们不仅可以发现关于活动自身的许多知识,诸如地域偏好,时间分布,活动之间的关系,还可以了解到用户的个人喜好,行为模式,进一步对用户进行个性化推荐。活动挖掘是一个有着诸多潜在应用的研究方向。

社交数据以其内容的丰富性和及时性,为日常活动的信息挖掘提供了便利,也带来许多挑战。本文确定了三个研究内容:(1)如果精确地从社交媒体中抽取出日常活动的概念表示?(2)针对用户发布的信息,如何抽取出相关活动及相关属性,如地点、时间、情感?(3)如何构建活动之间的关系?对于第一个问题,本文将概念抽取问题化归为分类问题,通过词的语义向量表示作为特征,通过解优化问题选取了训练数据,进行分类,取得了较好的分类精度。第二个问题中,我借助句法分析、情感分析、信息抽取等方法,实现了活动实例的抽取。第三个问题中,本文基于实例抽取的结果,着重研究了活动之间的序列关系。

此外,基于以上三点研究,本文构建了ActivityNet,实现了高效的活动检索,将本文的研究成果,进行直观的展示。

\section{进一步工作}
在社交数据中挖掘日常活动相关知识是一个新颖的问题,可以参考的工作不多,本文在这一课题上进行初步研究,取得了一定成果,但也存在一些局限性。在将来,本文的工作可以在以下方面进行改进:
\begin{enumerate}
	\item 考虑用户在社交网络中的社交关系。本文的工作中,用户是孤立的,没有考虑到用户之间的互动。而利用用户的好友关系和网络结构可以提供可以帮助我们建模更复杂的活动关系,例如发现用户和其好友共同的活动,以及他们在活动中的互动等。这将是一个有趣的研究问题。
	\item 对活动关系的建模比较粗糙。活动之间的关系是复杂多样的,本文主要关注活动的序列关系,而根据分类标准的不同,活动之间的关系是多样了。
	\item 本文现阶段的工作,仅在社交数据上进行。如果借助外部知识库,如地点、人物、书籍信息,将对活动抽取和关系建模有很大的帮助。
\end{enumerate}

希望本文的研究可以继续发展,更加深入地揭示社交数据与日常行为的关系,并投入实际应用。