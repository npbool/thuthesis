
%%% Local Variables:
%%% mode: latex
%%% TeX-master: t
%%% End:
\secretlevel{绝密} \secretyear{2100}

\ctitle{社交媒体中的活动挖掘}
% 根据自己的情况选,不用这样复杂
\makeatletter
\ifthu@bachelor\relax\else
  \ifthu@doctor
    \cdegree{工学博士}
  \else
    \ifthu@master
      \cdegree{工学硕士}
    \fi
  \fi
\fi
\makeatother

\cdepartment[计算机]{计算机科学与技术系}
\cmajor{计算机科学与技术}
\cauthor{王凝枰} 
\csupervisor{唐 杰 副教授}

% 日期自动生成,如果你要自己写就改这个cdate
%\cdate{\CJKdigits{\the\year}年\CJKnumber{\the\month}月}

% 博士后部分
% \cfirstdiscipline{计算机科学与技术}
% \cseconddiscipline{系统结构}
% \postdoctordate{2009年7月——2011年7月}

\etitle{Activity Mining in Social Media} 
% 这块比较复杂,需要分情况讨论:
% 1. 学术型硕士
%    \edegree:必须为Master of Arts或Master of Science(注意大小写)
%              “哲学、文学、历史学、法学、教育学、艺术学门类,公共管理学科
%               填写Master of Arts,其它填写Master of Science”
%    \emajor:“获得一级学科授权的学科填写一级学科名称,其它填写二级学科名称”
% 2. 专业型硕士
%    \edegree:“填写专业学位英文名称全称”
%    \emajor:“工程硕士填写工程领域,其它专业学位不填写此项”
% 3. 学术型博士
%    \edegree:Doctor of Philosophy(注意大小写)
%    \emajor:“获得一级学科授权的学科填写一级学科名称,其它填写二级学科名称”
% 4. 专业型博士
%    \edegree:“填写专业学位英文名称全称”
%    \emajor:不填写此项
\edegree{Bachelor of Engineering} 
\emajor{Computer Science and Technology} 
\eauthor{Ningping Wang} 
\esupervisor{Associate Professor Jie tang} 

% 这个日期也会自动生成,你要改么?
% \edate{December, 2005}

% 定义中英文摘要和关键字
\begin{cabstract}
随着社交媒体和移动互联网日益渗透用户的日常生活,用户越来越多的在社交媒体上发布和自己的日常生活相关的内容。社交媒体中活动信息的挖掘,在用户行为建模,个性化推荐等领域有着诸多潜在应用,但当前对于活动的挖掘仅有很少的工作。本文基于微博这一流行的社交网络,研究了在社交活动中进行活动挖掘的算法和框架。

本文将活动信息挖掘分解为多个子问题,首先从微博文本中抽取活动概念;其次,抽取活动相关属性,包括时间、地点、情感极性等,构成活动实例。基于概念抽取和实例抽取的结果,对活动的相关度和序列关系进行挖掘。

此外,本文构建了关于日常活动的知识库系统ActivityNet,对于用户的查询请求,通过地图、图表等形式,将上述工作得到的活动相关知识进行直观的展示。
\end{cabstract}

\ckeywords{活动挖掘,自然语言处理,信息抽取,社交媒体,知识库}

\begin{eabstract} 
As social media and mobile network gain popularity in people's everyday life, people are posting more and more activity-related content in social media. Activity mining in social media offers tons of potential applications, such as user behavior modeling and personalized recommendation. However, little work has been done in this area. Based on Weibo, one of the most popular social network in China, the thesis studies the algorithms and frameworks of activity mining.

This work treats activity mining as a series of sub-problems. First, extracting abstract activity concept in social content. Then, extracting activity attributes including place, time, sentiment polarity and building activity instances. Based and concept extraction and instance extraction result, mining similarity and seqential relations between activities.

Besides, a knowledge-base system Activity is built to demonstrate our work. In response to user's query, a visualized respresention of the knowledge about the activity is shown, using maps and charts.
\end{eabstract}

\ekeywords{Activity Mining, Natural Language Processing, Information Extraction, Social Media, Knowledge Base}
