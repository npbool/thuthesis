\chapter{相关工作}
本文实验的系统,与信息抽取、本体学习等领域有较强的相关性,同时用到了自然语言处理、机器学习的一些模型和方法。下面对本文主要的相关工作作简单的介绍。
\section{信息抽取}
信息抽取(Information Extraction)
\section{本体学习}
本体学习(Ontology Learning)

\section{知识库}
知识库是对人类知识的结构化表示,它在搜索、智能系统中有着日益重要的应用,Google,Microsoft以及国内百度、搜狗等互联网企业均有自己的知识图谱计划。下面是常见知识库概况的一个总结。

\begin{table}[!h]
\begin{tabular}[0.7\textwidth]{|l|p{2cm}|l|p{4cm}|}
\hline
名称 & 开发者 & 概念数量  & 概述 \\
\hline
SenticNet & 南洋理工大学、新加坡国立大学 & 14,244	& 情感词汇  \\
\hline
Freebase & 社区	& 1450	& 对不同领域的知名人物、地点、事物 \\
\hline
WordNet\cite{miller1995wordnet} & 普林斯顿大学 & 25,229 & 英语词汇的知识库,根据同义语将英语词汇进行组织,并且提供词汇之间的多种语义关系。 \\
\hline
WikiTaxonomy\cite{ponzetto2007deriving}	& HITS & 127,325 & 基于维基百科(Wikipedia)的语料, 将类别按照is-a关系构建为一个大规模分类系统 \\
\hline
\end{tabular}
\caption{现有知识库概况}
\label{table:knowledge_base}
\end{table}

从表\ref{table:knowledge_base}中可以看到,现有的知识库种类繁多,但主要关注词汇,客观实体如人物、地点、物体的关系建模,没有对人类日常活动给予足够的关注。希望我的工作能在这方面对现有的知识库的一个补充。

\section{情感分析Sentiment Analysis}
我们希望了解用户参与
