\chapter{ActivityNet系统设计}

\section{设计目标}
通过之前的工作,本文从社交媒体中抽取了活动概念、地点、情感等属性,并挖掘了活动间的相关性和序列信息。基于这些结果,本文构建了一个可视化系统ActivityNet。系统希望实现以下目标:
\begin{itemize}
\item 高效的活动检索
\item 良好的可视化界面
\item 提供用户反馈机制,纠正错误分类
\end{itemize}

\section{底层架构}
为了系统的高效性,我们选择SAE作为系统底层架构。SAE全称为Social Analytic Engine,即社交网络分析引擎,是清华大学计算机系KEG研究组开发的一套社交网络分析平台,架构图见图\ref{fig:sae_arch},提供了以下功能:
\begin{enumerate}
\item 存储和快速检索极大规模社交网络的数据
\item 提供最新的社交网络分析算法和机器学习算法,如话题模型(Topic Model), 影响力最大化模型(Influence Maximization)等
\item 提供通用的网络分析引擎和机器学习引擎.
\end{enumerate}

\begin{figure}[!h]
\centering
\includegraphics[width=0.7\textwidth]{sae.png}
\caption{SAE基本结构}
\label{fig:sae_arch}
\end{figure}

在ActivityNet中,我们主要用到SAE的存储和索引机制。由于SAE主要用于处理社交网络数据,因此图是SAE主要的存储模型。本文将活动以及之间的关系抽象为图结构进行存储。为了检索的高效,对所有活动名称建立了索引。同时,由于相似活动搜索是基于语义相似度的,本文也对词向量使用KD-Tree进行索引。

\section{功能实现}
ActivityNet以Web站点的形式提供服务,基于成熟的前端框架$Twitter Bootstrap$,简化了开发工作。下面我们以查询``吃烤鸭''为例,介绍系统中各个功能。

\subsection{首页设计}
ActivityNet的首页如图\ref{fig:frontpage}。

\begin{figure}[!h]
\centering
\includegraphics[width=\textwidth]{frontpage.png}
\caption{Activity首页设计}
\label{fig:frontpage}
\end{figure}

首页提供以下功能
\begin{enumerate}
\item 活动搜索。用户可以自行输入自己希望查询的活动
\item 城市热门活动推荐。我们计算出北京、上海、广州的热门活动,在首页进行推荐。
\item 全网热门活动。基于活动实例抽取结果,我们选择了18个最热门的活动,在首页底部呈现。活动对应的图片是以活动本身为关键词构建GET请求,在百度图片搜索抓取第一张图。自动抓取的个别活动图片不很理想。我们在自动抓取图片后,人工进行一些调整。
\end{enumerate}

\subsection{活动搜索}
用户输入一个查询后,可以进入此活动查询结果的二级页面。在此,输入``吃烤鸭'',查询结果如图\ref{fig:query_kaoya}

\begin{figure}[h]
\centering
\includegraphics[width=\textwidth]{kaoya_new.png}
\caption{查询结果}
\label{fig:query_kaoya}
\end{figure}

在页面右侧,我们列出了和查询相关的前10个活动,以相关性排序;其次,利用序列关系抽取的结果,我们列出了此活动经常发生的后续活动。烤鸭是北京的特色菜品,``相关活动''中,我们列出的``炸酱面'',``涮羊肉'',``麻辣烫''也都是北京的特色菜。在序列关系中,ActivityNet提供了``散步'',``回家'',``游泳'',``探索城市'',``逛街'',``逛超市''等活动,是比较合理的。由于概念的抽取难以做到完全的精准,ActivityNet的结果中有一些噪声,为此,我们提供用户反馈机制。如果用户发现返回的结果不是活动,鼠标移上之后会呈现``$\times$''按钮,点击后会向系统报告错误。

\subsection{地点、情感、时间分布}
基于地点抽取和情感分类的结果,本文利用Google Map API将地理分布和情感分布可视化。上图是用户在参加活动是情感状态的分布,一个圆点表示一个活动实例,红色为正面情感,蓝色为负面情感,绿色为中性。下图是这项活动的在不同地域的分布情况,以不同颜色的弥散圆表示,红色的强度表示和这项活动相关的活动相关的微博数目。可以看到,``吃烤鸭''这项活动,在北京最为流行;在上海、成都、重庆等地也有着较多的分布。

进一步分析了这项活动发生的POI的类型。从POIType的饼状图中可以看到,``吃烤鸭''最常发生在``北京菜''这类POI中,其次还有为``中餐厅''。而在时间分布中,``吃烤鸭''最常发生的时间为6点。

\section{本章小结}
本章从底层架构到界面设计介绍了基于活动信息挖掘的可视化系统ActivityNet。ActivityNet基于SAE(Social Analytic Engined)实现了高效的存储和检索功能,并利用gchart和Google Map API实现信息可视化,实现了较好的用户体验。由于方法的局限,ActivityNet呈现的结果有一些噪声,系统中提供了用户反馈机制,使得不正确的结果可以及时更正。

