\chapter{绪论}
\section{研究背景和意义}
以Facebook,Twitter,以及国内新浪微博、人人网为代表的社交网络,在过去几年中取得了巨大的成功。Facebook全球已经有超过10亿用户,新浪微博也有超过6000万用户,据报道,在美国,人们16\%的上网时间停留在Facebook上,超过了传统搜索引擎Google(10\%)。

而对用户线下行为的建模,也是我们十分关注的话题,活动(Activity)信息的挖掘,是比较新,但十分令人激动的领域。现有的系统,如大众点评,Yelp等,通过搜集用户对商家的点评和反馈,可以根据用户所在地区和时间在饮食、旅游、休闲娱乐等方面对用户提供商家的推荐。然而这些系统依赖于用户自发的点评行为,局限在对商家的评价,没有对活动之间的关系进行建模。

而社交媒体,则是用户生活自然的反应。我们希望利用用户在社交网络上的发布的信息,包括文本、签到等,帮助我们对用户的线下行为模式和喜好建模。

在社交网络中
构建知识库,推断活动的关系,以及活动的自身属性。这一点是现有知识库缺少的。
了解用户行为习惯,对行为进行建模。
个性化的活动推荐。比如,用户进入一个兴趣点(POI, point of interest),我们可以根据时间、用户之间在社交网络中发布的消息,预测用户可能感兴趣的活动。

\section{问题和挑战}
从freestyle的文本中抽取活动概念和实例
利用有限的文本信息,对活动的关系进行推断
借助社交网络的关系结构,

\section{本文主要工作}
本文目标在于建立一个关于人类活动的知识库系统ActivityNet。

\subsection{概念语义建模和分类}
为了从微博文本中抽取活动概念,本文利用自然语言处理的神经网络语言模型训练得到短语的向量表示,同时提出最优化标注集合的算法,使得在训练数据有限的条件下优化分类器的性能。同时建立用户反馈机制,纠正分类错误的样本,实现在线学习,以提高模型精度。
\subsection{活动实例抽取}

\subsection{概念关系建模}

\section{论文组织}
本文的章节安排如图\ref{fig:organ}所示。

{\heiti 第二章} 介绍了活动挖掘的相关研究,包括信息抽取,本体学习,知识库等。

{\heiti 第三章} 介绍了基于基于神经网络语言模型和分类算法的概念抽取模型,并对结果进行了分析。

{\heiti 第四章} 介绍了基于规则和语法解析工具对活动实例的抽取方法。

{\heiti 第六章} 介绍了根据对活动概念、实例的抽取结果,进行知识发现,构建层次概念模型等方法。

{\heiti 第七章} 介绍了本文研究中构建的系统ActivityNet,从系统的基础架构和可视化方面对该系统进行详细的介绍。

{\heiti 第八章} 对本文的工作进行总结,并提出进一步的研究方向。

\begin{figure}[!h]
\caption{论文组织}
\label{fig:organ}
\begin{center}
\includegraphics[width=0.5\textwidth]{organ.png}
\end{center}
\end{figure}

\chapter{相关工作}
本文实验的系统,与信息抽取、本体学习等领域有较强的相关性,同时用到了自然语言处理、机器学习的一些模型和方法。下面对本文主要的相关工作作简单的介绍。
\section{信息抽取}
信息抽取(Information Extraction)
\section{本体学习}
本体学习(Ontology Learning)

\section{知识库}
知识库是对人类知识的结构化表示,它在搜索、智能系统中有着日益重要的应用,Google,Microsoft以及国内百度、搜狗等互联网企业均有自己的知识图谱计划。下面是常见知识库概况的一个总结。

\begin{table}[!h]
\begin{tabular}[0.7\textwidth]{|l|p{2cm}|l|p{4cm}|}
\hline
名称 & 开发者 & 概念数量  & 概述 \\
\hline
SenticNet & 南洋理工大学、新加坡国立大学 & 14,244	& 情感词汇  \\
\hline
Freebase & 社区	& 1450	& 对不同领域的知名人物、地点、事物 \\
\hline
WordNet\cite{miller1995wordnet} & 普林斯顿大学 & 25,229 & 英语词汇的知识库,根据同义语将英语词汇进行组织,并且提供词汇之间的多种语义关系。 \\
\hline
WikiTaxonomy\cite{ponzetto2007deriving}	& HITS & 127,325 & 基于维基百科(Wikipedia)的语料, 将类别按照is-a关系构建为一个大规模分类系统 \\
\hline
\end{tabular}
\caption{现有知识库概况}
\label{table:knowledge_base}
\end{table}

从表\ref{table:knowledge_base}中可以看到,现有的知识库种类繁多,但主要关注词汇,客观实体如人物、地点、物体的关系建模,没有对人类日常活动给予足够的关注。希望我的工作能在这方面对现有的知识库的一个补充。
