\chapter{活动关系抽取及知识发现}
\section{构建活动层次}
如何从一系列概念中挖掘出层次关系,目前有一些工作,例如KDD2013年的一篇文章\ref{wang2013phrase},是在科技文献中构建研究领域的层次关系,本文在其基础上进行扩充和修改,以适应活动层次挖掘的任务。

\subsection{算法概述}


\section{序列关系挖掘}
与通常知识库系统不同,本文还关注挖掘活动直接的序列关系(follow-up),即用户在参加一项活动后,通常进行的下一项活动是什么。这一点有助于帮助我们对用户行为进行建模,进行活动的推荐。通过之前的工作,我们已经在微博语料中抽取出活动的事例,序列关系的抽取可以在这个基础上进行。序列关系的强弱包含两个方面
\begin{enumerate}
\item 用户进行活动$c_i$后,在时间窗口$T$内,进行活动$c_j$的概率
\item 用户进行活动$c_i$和$c_j$之间的期望时间
\end{enumerate}
这两个方面缺一不可。由于用户行为的复杂性,第一项可以对噪声进行抑制,避免个别用户随机行为的影响,使挖掘出的活动有较高的置信度;第二项表示两项活动间隔的时间越短,它们的序列关系越密切。基于这两个考虑,我们对问题定义如下:

\begin{problem}[序列关系挖掘]
给出
\begin{itemize}
\item 活动概念集合$C={c_i},i=1,2,\ldots,N^c$,$N^c$为活动概念的数量。
\item 用户集合$U=\{u_i\},i=1,2,\ldots,N^u$,$N^u$为用户数量
\item 活动实例集合$A = {A_i}$。对每个用户$u_i$,有其参加活动实例的集合$A_i = \{a_{ij}\}, j=1,2,\ldots,N_i^a$,$N_i^a$是用户$u_i$在给定微博语料中参与活动实例的个数,每个活动实例$a_{ij}$是活动概念、时间、地点、情感极性的四元组,即$(c,t,p,s)$。
\end{itemize}
求在给定时间窗口$T$内,
\begin{enumerate}
\item 用户进行活动概念$c_i$后进行活动$c_j$的概率$P(c_i|c_j,T)$
\item 在$P(c_i|c_j,T)>\lambda$的条件下,$E(t_{c_j} - t_{c_i})$
\end{enumerate}
\end{problem}

根据问题的定义,可以得到算法如下:

\begin{algorithm}
  \caption{序列关系挖掘 }
  \KwIn{活动实例集合$C$, 用户集合$U$, 活动实例集合$A$, 阈值$\lambda$}
  \KwOut{对每个活动概念$c_i\in C$, 随后可能发生活动的序列$l_i$}
  
\end{algorithm}


\section{时间、地域分布挖掘}
